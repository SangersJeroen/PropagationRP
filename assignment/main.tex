%%%%%%%%%%%%%%%%%%%%%%%%%%%%%%%%%%%%%%%%%
% Lachaise Assignment
% LaTeX Template
% Version 1.0 (26/6/2018)
%
% This template originates from:
% http://www.LaTeXTemplates.com
%
% Authors:
% Marion Lachaise & François Févotte
% Vel (vel@LaTeXTemplates.com)
%
% License:
% CC BY-NC-SA 3.0 (http://creativecommons.org/licenses/by-nc-sa/3.0/)
%
%%%%%%%%%%%%%%%%%%%%%%%%%%%%%%%%%%%%%%%%%

%----------------------------------------------------------------------------------------
%	PACKAGES AND OTHER DOCUMENT CONFIGURATIONS
%----------------------------------------------------------------------------------------

\documentclass{article}

\input{structure.tex} % Include the file specifying the document structure and custom commands
%Als we extra packages willen laden kan dat het best in structure.tex

% Define matices for pretty functions
\def\r{
	\begin{pmatrix}
		x \\
		y \\
		z 
	\end{pmatrix}}
	
\def\dr{
	\begin{pmatrix}
		\dot{x} \\
		\dot{y} \\
		\dot{z} 
	\end{pmatrix}}
	
\def\Nabla{
	\begin{pmatrix}
		\frac{\partial}{\partial {x}} \\
		\frac{\partial}{\partial {y}} \\
		\frac{\partial}{\partial {z}} 
	\end{pmatrix}}
	
\def\dNabla{
	\begin{pmatrix}
		\frac{\partial}{\partial \dot{x}} \\
		\frac{\partial}{\partial \dot{y}} \\
		\frac{\partial}{\partial \dot{z}} 
	\end{pmatrix}}
	
	
\begin{document}

\section{Principle of Fermat}

\textit{\underline{Question:} Proof that \ref{eq_1.3} and \ref{eq_1.4} can be reduced to equation \ref{eq_1.6}.} \\

\begin{equation}
	\label{eq_1.3}
	L[x,y,z,\dot{X},\dot{y},\dot{z}] = n(x,y,z)\sqrt{\dot{x}^2+\dot{y}^2+\dot{z}^2}
\end{equation}

\begin{equation}
	\label{eq_1.4}
	\frac{d}{dt} \frac{\partial L}{\partial \dot{x}} - \frac{\partial L}{\partial x} = 0, \frac{d}{dt} \frac{\partial L}{\partial \dot{y}} - \frac{\partial L}{\partial y} = 0, \frac{d}{dt} \frac{\partial L}{\partial \dot{z}} - \frac{\partial L}{\partial z} = 0
\end{equation}

\begin{equation}
	\label{eq_1.6}
	\frac{d}{ds} \left[ n \frac{d \vec{r}}{ds} \right] = \vec{\nabla} n
\end{equation}

\textit{\underline{Answer:}}\\
\\
First noting that $ds$ is a small element of distance travelled. Therefore taking into account the $x$, $y$ and $z$ direction, $ds$ is given by: \\

\begin{equation}
	ds = \sqrt{{dx}^2+{dy}^2+{dz}^2}
\end{equation}

A small distance travelled in a trivial direction, lets say $dx$, can be approximated by as $dx = dt \cdot \dot{x}$. Therefore $ds$ can be rewritten as: \\

\begin{equation}
	ds = dt \sqrt{\dot{x}^2+\dot{y}^2+\dot{z}^2}
\end{equation}

Rewriting gives:

\begin{equation}
	\label{eq_dsdt}
	\sqrt{\dot{x}^2+\dot{y}^2+\dot{z}^2} = \frac{ds}{dt}
\end{equation}

If we combine the equations in equation \ref{eq_1.4} in vector notation we get: \\

\begin{equation}
	\frac{d}{dt} \dNabla L - \nabla L = 0
\end{equation}

Rewriting and filling in equation \ref{eq_1.3} gives:

\begin{equation}
	\frac{d}{dt} \left[ \dNabla n(x,y,z)\sqrt{\dot{x}^2+\dot{y}^2+\dot{z}^2} \right] = \Nabla \left[ n(x,y,z)\sqrt{\dot{x}^2+\dot{y}^2+\dot{z}^2} \right]
\end{equation}

\begin{equation}
	\frac{d}{dt}  \frac{n \cdot \dot{\vec{r}} }{\sqrt{\dot{x}^2+\dot{y}^2+\dot{z}^2}}   = \vec{\nabla} n \; \sqrt{\dot{x}^2+\dot{y}^2+\dot{z}^2}
\end{equation}

Using equation \ref{eq_dsdt} to to replace the $\sqrt{\dot{x}^2+\dot{y}^2+\dot{z}^2}$ and rewriting the $\dot{\vec{r}}$ vector gives the following: \\

\begin{equation}
	\frac{d}{dt} \frac{n \cdot \dot{\vec{r}}}{\frac{ds}{dt}}  = \vec{\nabla} n \: \frac{ds}{dt}
\end{equation}

\begin{equation}
	\frac{d}{dt}  \frac{n \cdot \frac{d}{dt} \vec{r}}{\frac{ds}{dt}}   =  \vec{\nabla} n \: \frac{ds}{dt}
\end{equation}

Rewriting yields the equation that was to be proved: \\

\begin{equation}
	\frac{d}{ds} \left[ n \frac{d \vec{r}}{ds} \right] = \vec{\nabla} n
\end{equation}

\section{Application}
\subsection{Homogeneous medium}

\textit{\underline{Question:} Using equation \ref{eq_1.6}, show how light is travelling in a homogeneous medium.}\\
\\
\textit{\underline{Answer:}} \\
\\
Equation \ref{eq_1.6} can be rewritten using the chain rule: \\

\begin{equation}
	\frac{d \vec{r}}{ds} \frac{d}{ds} n + n \frac{d^2 \vec{r}}{ds^2} = \Nabla n
\end{equation}

Note that for a homogeneous medium, the index of refraction, $n$, is constant. Therefore $\frac{dn}{ds} = 0$, $\frac{\partial n}{\partial x} = 0$, $\frac{\partial n}{\partial y} = 0$ and $\frac{\partial n}{\partial z} = 0$. Using this in the previous equation yields: \\

\begin{equation}
	 n \frac{d^2 \vec{r}}{ds^2} = \vec{0}
\end{equation}

\begin{equation}
	 \frac{d^2 \vec{r}}{ds^2} = \vec{0}
\end{equation}


This implies that the direction and velocity of the light is not changed as the light travels through the medium. Therefore, it travels in a straight line with a constant velocity of $ v = c/n $. \\


\end{document}
