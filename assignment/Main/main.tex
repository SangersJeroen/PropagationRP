%%%%%%%%%%%%%%%%%%%%%%%%%%%%%%%%%%%%%%%%%
% Lachaise Assignment
% LaTeX Template
% Version 1.0 (26/6/2018)
%
% This template originates from:
% http://www.LaTeXTemplates.com
%
% Authors:
% Marion Lachaise & François Févotte
% Vel (vel@LaTeXTemplates.com)
%
% License:
% CC BY-NC-SA 3.0 (http://creativecommons.org/licenses/by-nc-sa/3.0/)
%
%%%%%%%%%%%%%%%%%%%%%%%%%%%%%%%%%%%%%%%%%

%----------------------------------------------------------------------------------------
%	PACKAGES AND OTHER DOCUMENT CONFIGURATIONS
%----------------------------------------------------------------------------------------

\documentclass{article}

\input{structure.tex} % Include the file specifying the document structure and custom commands
%Als we extra packages willen laden kan dat het best in structure.tex


% Define matices for pretty functions
\def\r{
	\begin{pmatrix}
		x \\
		y \\
		z 
	\end{pmatrix}}
	
\def\dr{
	\begin{pmatrix}
		\dot{x} \\
		\dot{y} \\
		\dot{z} 
	\end{pmatrix}}
	
\def\Nabla{
	\begin{pmatrix}
		\frac{\partial}{\partial {x}} \\
		\frac{\partial}{\partial {y}} \\
		\frac{\partial}{\partial {z}} 
	\end{pmatrix}}
	
\def\dNabla{
	\begin{pmatrix}
		\frac{\partial}{\partial \dot{x}} \\
		\frac{\partial}{\partial \dot{y}} \\
		\frac{\partial}{\partial \dot{z}} 
	\end{pmatrix}}
	
	
\begin{document}

\section{Principle of Fermat}

\textit{\underline{Question:} Proof that equation \ref{eq_1.3} and \ref{eq_1.4} can be reduced to equation \ref{eq_1.6}.} \\

\begin{equation}
	\label{eq_1.3}
	L[x,y,z,\dot{X},\dot{y},\dot{z}] = n(x,y,z)\sqrt{\dot{x}^2+\dot{y}^2+\dot{z}^2}
\end{equation}

\begin{equation}
	\label{eq_1.4}
	\frac{d}{dt} \frac{\partial L}{\partial \dot{x}} - \frac{\partial L}{\partial x} = 0, \frac{d}{dt} \frac{\partial L}{\partial \dot{y}} - \frac{\partial L}{\partial y} = 0, \frac{d}{dt} \frac{\partial L}{\partial \dot{z}} - \frac{\partial L}{\partial z} = 0
\end{equation}

\begin{equation}
	\label{eq_1.6}
	\frac{d}{ds} \left[ n \frac{d \vec{r}}{ds} \right] = \vec{\nabla} n
\end{equation} \\
\textit{\underline{Answer:}}\\
\\
First noting that $ds$ is a small element of distance travelled. Therefore taking into account the $x$, $y$ and $z$ direction, $ds$ is given by: \\

\begin{equation}
	ds = \sqrt{{dx}^2+{dy}^2+{dz}^2}
\end{equation}

A small distance travelled in a trivial direction, lets say $dx$, can be approximated by as $dx = dt \cdot \dot{x}$. Therefore $ds$ can be rewritten as: \\

\begin{equation}
	ds = dt \sqrt{\dot{x}^2+\dot{y}^2+\dot{z}^2}
\end{equation}

Rewriting gives:

\begin{equation}
	\label{eq_dsdt}
	\sqrt{\dot{x}^2+\dot{y}^2+\dot{z}^2} = \frac{ds}{dt}
\end{equation}

If we combine the equations in equation \ref{eq_1.4} in vector notation we get: \\

\begin{equation}
	\frac{d}{dt} \dNabla L - \nabla L = 0
\end{equation}

Rewriting and filling in equation \ref{eq_1.3} gives:

\begin{equation}
	\frac{d}{dt} \left[ \dNabla n(x,y,z)\sqrt{\dot{x}^2+\dot{y}^2+\dot{z}^2} \right] = \Nabla \left[ n(x,y,z)\sqrt{\dot{x}^2+\dot{y}^2+\dot{z}^2} \right]
\end{equation}

\begin{equation}
	\frac{d}{dt}  \frac{n \cdot \dot{\vec{r}} }{\sqrt{\dot{x}^2+\dot{y}^2+\dot{z}^2}}   = \vec{\nabla} n \; \sqrt{\dot{x}^2+\dot{y}^2+\dot{z}^2}
\end{equation}

Using equation \ref{eq_dsdt} to to replace the $\sqrt{\dot{x}^2+\dot{y}^2+\dot{z}^2}$ and rewriting the $\dot{\vec{r}}$ vector gives the following: \\

\begin{equation}
	\frac{d}{dt} \frac{n \cdot \dot{\vec{r}}}{\frac{ds}{dt}}  = \vec{\nabla} n \: \frac{ds}{dt}
\end{equation}

\begin{equation}
	\frac{d}{dt}  \frac{n \cdot \frac{d}{dt} \vec{r}}{\frac{ds}{dt}}   =  \vec{\nabla} n \: \frac{ds}{dt}
\end{equation}

Rewriting yields the equation that was to be proved: \\

\begin{equation}
	\frac{d}{ds} \left[ n \frac{d \vec{r}}{ds} \right] = \vec{\nabla} n
\end{equation}

\section{Application}
\subsection{Homogeneous medium}

\textit{\underline{Question:} Using equation \ref{eq_1.6}, show how light is travelling in a homogeneous medium.}\\
\\
\textit{\underline{Answer:}} \\
\\
Equation \ref{eq_1.6} can be rewritten using the chain rule: \\

\begin{equation}
	\frac{d \vec{r}}{ds} \frac{d}{ds} n + n \frac{d^2 \vec{r}}{ds^2} = \Nabla n
\end{equation} \\

Note that for a homogeneous medium, the index of refraction, $n$, is constant. Therefore $\frac{dn}{ds} = 0$, $\frac{\partial n}{\partial x} = 0$, $\frac{\partial n}{\partial y} = 0$ and $\frac{\partial n}{\partial z} = 0$. Using this in the previous equation yields: \\

\begin{equation}
	 n \frac{d^2 \vec{r}}{ds^2} = \vec{0}
\end{equation}

\begin{equation}
	 \frac{d^2 \vec{r}}{ds^2} = \vec{0}
\end{equation} \\


This implies that the direction and velocity of the light is not changed as the light travels through the medium. Therefore, it travels in a straight line with a constant velocity of $ v = c/n $. \\

\subsection{Snell-Descartes Law}

\textit{\underline{Question:} Express first geometrically and then analytically Snell and Descartes law of reflection and transmission of the light at the interface between two media of different index of refraction $n_1$ and $n_2$, using the Principle of Fermat and equation \ref{eq_1.6}.}\\
\\
\textit{\underline{Answer:}} \\
\\

\subsubsection{Geometrical}
The speed of light in a medium is inversely proportional to the refractive index. Therefore, the shortest path (in distance) between two points in materials with different refractive indices is not always the fastest (in time). This phenomenon is nicely described by a 2-dimensional analogy of a beach (see figure \ref{fig_snell_empty}). The maximum speed on foot on beach is significantly higher than the maximum swimming speed in the water. So if somebody would need to get from a point A on the beach to a point B in the water, the direct route from A to B (dashed line in figure \ref{fig_snell_empty}) would intuitively be slower than the path with a shorter swimming distance (solid line in figure \ref{fig_snell_empty}). 

\begin{figure}[h!]
	\centering
	\includegraphics[width=8cm]{afbeeldingen/snell_diagram_leeg.jpg}
	\caption{Diagram of a 2-dimensional beach analogy of the interface between two media with different index of refraction. The upper-half corresponds to the beach and the lower-half to the sea. The dashed line corresponds to the direct route between point A and B with the shortest distance. The solid line corresponds to a route that is intuitively faster than the direct route.}
	\label{fig_snell_empty}
\end{figure}

It is possible to calculate the fastest route between point A and B If we add the parameters $v_1$, $v_2$, $\theta _1$, $\theta _2$, $a$, $b$, $c$ and $d$ which corresponds respectively to the propagation speed on the beach, the propagation speed in the water, the angle of the path on the beach with the normal, the angle of the path in the water with the normal and distances which can be seen in figure \ref{fig_snell_full}. 

\begin{figure}[h!]
	\centering
	\includegraphics[width=8cm]{afbeeldingen/snell_diagram_vol.jpg}
	\caption{Diagram of beach analogy parameters $v_1$, $v_2$, $\theta _1$, $\theta _2$, $a$, $b$, $c$ and $d$. These correspond respectively to the propagation speed on the beach, the propagation speed in the water, the angle of the path on the beach with the normal, the angle of the path in the water with the normal and distances which can be seen in the diagram.}
	\label{fig_snell_full}
\end{figure} 

The time it takes to travel from point A to B, $t$, can easily be found dividing the path on the beach and the water, respectively $l_{beach}$ and $l_{water}$ by the corresponding speed: 

\begin{equation}
	t = l_{beach}/v_1 + l_{water}/v_2
\end{equation} 

Using the pythagoras theorem we find: 

\begin{equation}
	t = \sqrt{a^2 + c^2}/v_1 + \sqrt{b^2 + (d-c)^2}/v_2
	\label{eq_t}
\end{equation} 

If there is a fastest path, there should be an optimum value for $c$ for which $dt/dc = 0$. Therefore, applying the principle of Fermat to equation \ref{eq_t} leads to the following: 

\begin{equation}
	0 = \frac{c}{v_1 \sqrt{a^2 + c^2}} + \frac{c-d}{v_2 \sqrt{b^2 + (d-c)^2}}
\end{equation}

Using the trigonometric identity $sin(\theta) = (adjacent side)/(diagonal side)$ for right-angled triangle we obtain: 

\begin{equation}
	0 =  sin(\theta _1)/v_1 - sin(\theta _2)/v_2
\end{equation}

If we rewrite this and use the fact that the speed of light in a medium is given by $v = c/n$ we obtain Snell-Descartes law: 

\begin{equation}
	n_1 \; sin(\theta _1) = n_2 \; sin(\theta _2) 
\end{equation}

This equation basically tells us, that for a interface to a higher refractive index, so where the light slows down, the light bends to the normal. 

\subsubsection{Analytical}

For the analytical derivation of Snell-Descartes law we will use a similar diagram as in the geometrical derivation with a coordinate system added as in figure \ref{fig_snell_analytical}.

\begin{figure}[h!]
\centering
  \begin{subfigure}[b]{0.6\textwidth}
    \includegraphics[width=\textwidth]{afbeeldingen/snell_analytical.jpg}
	\caption{Diagram of the path of light at the interface of two media with different refractive indices $n_1$ and $n_2$. $\theta _1$ and $\theta _2$ correspond to the angle with the normal.}
	\label{fig_snell_analytical}
  \end{subfigure}
  %
  \begin{subfigure}[b]{0.2\textwidth}
    \includegraphics[width=\textwidth]{afbeeldingen/ds.jpg}
	\caption{$ds$ in relation to $dx$ and $dy$.}
	\label{fig_ds}
  \end{subfigure}
  \caption{}
\end{figure}


If we write equation \ref{eq_1.6} for only the x-component and use the fact that $n$ is independent of $x$ in our diagram, we get the following:

\begin{equation}
	\frac{d}{ds} \left[ n \frac{d x}{ds} \right] = \frac{d n}{dx}
\end{equation}

\begin{equation}
	\frac{d}{ds} \left[ n \frac{d x}{ds} \right] = 0
\end{equation}

The $ds$ in the latter equation is defined as in figure \ref{fig_ds}. If we keep a fixed $ds$ for both media we obtain the following equality:

\begin{equation}
	\frac{d}{ds} \left[ n_1 \frac{d x_1}{ds} \right] = \frac{d}{ds} \left[ n_2 \frac{d x_2}{ds} \right]
\end{equation}

Integrating both sides with respect to $ds$ and using the trigonometric identity, $sin(\theta)= dx/ds$, yields the Snell-Descartes law:

\begin{equation}
	n_1 \frac{d x_1}{ds}=n_2 \frac{d x_2}{ds}
\end{equation}

\begin{equation}
	n_1 \; sin(\theta _1) = n_2 \; sin(\theta _2)
	\label{eq_snell}
\end{equation}


\subsection{Optical fibre}

\subsubsection{}

\textit{\underline{Question:} If we assume an incoming beam in the $\vartheta _{x,y}$ plane at start crossing $\vartheta _x$ with an angle $\theta _0$. Show that the light beam will stay in this plane and that, by solving the Euler-Lagrange equation \ref{eq_1.6}, the equation for the beam path can be written as:}\\
\begin{equation}
	n \; sin(i) = a
	\label{eq_2.1}
\end{equation}
\\
\textit{\underline{Answer:}} \\
\\

For this question it is assumed that the optical fibre is cylindrical with its axis $\vartheta _x$ (see figure \ref{fig_fibre_3d}). The incident-surface will therefore be in the $\vartheta _{y,z}$ plane.\\

\begin{figure}[h!]
	\centering
	\includegraphics[width=7cm]{afbeeldingen/fibre_3d.jpg}
	\caption{3-dimensional diagram of fibre with incoming light beam making an angle $\theta _0$ with the $\vartheta _x$ axis.}
	\label{fig_fibre_3d}
\end{figure}

When the incoming beam is in the $\vartheta _{x,y}$ plane, the angle with the normal is $\theta _0$ with $\vartheta _x$ in the $\vartheta _y$ direction and $\theta _z = 0$ in the $\vartheta _z$ direction. If we use Snell-Descartes law (equation \ref{eq_snell}) for $\theta _z$ with $\theta _{z,1} = 0$ we obtain $\theta _{z,2} = k \cdot \pi [rad]$ with $k = -1,0,1,2,...$. Therefore, the beam of light will stay in the $\vartheta _{x,y}$ plane at the interface when entering the optical fibre. \\
It is the property of a cylinder that the outer surface is always perpendicular with the radius. For this light beam. Since the light beam is travelling in the $\vartheta _{x,y}$ direction, the outer surface it reaches will be in the $\vartheta _{x,z}$ plane. The angle with the surface normal in the $\vartheta _z$ axis is still zero. From the internal reflection it follows that the angle in  the $\vartheta _z$ axis will remain zero. Therefore, for every reflection at the outer surface, the beam will stay in the $\vartheta _{x,y}$ direction.\\
\\ 
Since we know that the beam will stay in the $\vartheta _{x,y}$ plane, the problem can  be simplified to a 2-dimensional problem (see figure \ref{fig_fibre_2d}). The infinitely small segment $ds$ is related to $ds$, $dy$, $dx$ and $i$ as in figure \ref{fig_fibre_2d}. 

\begin{figure}[h!]
	\centering
	\includegraphics[width = 5cm]{afbeeldingen/fibre_2d.jpg}
	\caption{2-dimensional fibre including the relation between $ds$, $ds$, $dy$, $dx$ and $i$.}
	\label{fig_fibre_2d}
\end{figure}

We can write equtaion \ref{eq_1.6} for only the x-component and use that $n$ is independent of $x$:

\begin{equation}
	\frac{d}{ds} \left[ n \frac{d x}{ds} \right] = \frac{d n}{dx}
\end{equation}

\begin{equation}
	\frac{d}{ds} \left[ n \frac{d x}{ds} \right] = 0
\end{equation}

If we now integrate with respect to $ds$ on both sides we get the following:

\begin{equation}
	n \frac{d x}{ds} = a
\end{equation}

With $a$ an arbitrary constant. Using the trigonometric identity $sin(i)= dx/ds$ we obtain the requested equation:

\begin{equation}
	n \; sin(i) = a
\end{equation}

\subsubsection{}

\textit{\underline{Question:} Solve equation \ref{eq_2.1} for $n(r) = n_0 \sqrt{1 - \alpha ^2 r^2}$ with $\alpha < 1/r$ a constant and $r$ the radius of the fibre.You should end up with an analytical function.}
\\
\textit{\underline{Answer:}} \\
\\



From figure \ref{fig_fibre_2d} and trigonometric identities it follows that $ sin(i) = dx/ds$. Using the Pythagoras theorem and filling this in in equation \ref{eq_2.1} we get:

\begin{equation}
	n  \frac{dx}{\sqrt{dr^2 + dx^2}} = a
\end{equation}

Filling in $n(r)$ and some rewriting results in the following:

\begin{equation}
	n_0 \sqrt{1 - \alpha ^2 r^2}  \frac{dx}{\sqrt{dr^2 + dx^2}} = a
\end{equation}

\begin{equation}
	n_0^2 (1 - \alpha ^2 r^2)  \frac{dx^2}{dr^2 + dx^2} = a^2
\end{equation}

\begin{equation}
	n_0^2 (1 - \alpha ^2 r^2)  dx^2 = a^2 (dr^2 + dx^2)
\end{equation}

\begin{equation}
	(n_0 ^2 - n_0^2 \alpha ^2 r^2 - a^2)  dx^2 = a^2 dr^2
\end{equation}

\begin{equation}
	dx^2 =\frac{a^2 dr^2}{n_0 ^2 - n_0^2 \alpha ^2 r^2 - a^2}
\end{equation}

\begin{equation}
	dx =\frac{a \; dr}{\sqrt{n_0 ^2 - n_0^2 \alpha ^2 r^2 - a^2}}
\end{equation}

If we now integrate both sides we will get:

\begin{equation}
	\int dx = \int \frac{a}{\sqrt{n_0 ^2 - n_0^2 \alpha ^2 r^2 - a^2}} dr
\end{equation}

\begin{equation}
	x = \frac{a}{\sqrt{n_0^2 - a^2}} \int \frac{1}{1 - \left( \frac{r n_0 \alpha}{\sqrt{n_0^2-a^2}} \right) ^2} dr
\end{equation}

To solve this integral we need the following substitution:

\begin{equation}
	 \frac{r n_0 \alpha}{\sqrt{n_0^2-a^2}} = sin(u)
\end{equation}

\begin{equation}
	 dr = \frac{\sqrt{n_0^2-a^2}}{n_0 \alpha} cos(u) du
\end{equation} 

\begin{equation}
	 u = arcsin \left( \frac{r n_0 \alpha}{\sqrt{n_0^2-a^2}} \right)
\end{equation}

Using the substitution we get:

\begin{equation}
	x = \frac{a}{n_0 \; \alpha} \int \frac{\cos (u)}{\sqrt{1 - \sin^2 (u)}} du
\end{equation}

If we use that $1 - \sin^2 (x) = \cos^2 (x)$, we obtain:

\begin{equation}
	x = \frac{a}{n_0 \; \alpha} \int \frac{\cos (u)}{\sqrt{\cos^2 (u)}} du
\end{equation}

\begin{equation}
	x = \frac{a}{n_0 \; \alpha} \int du
\end{equation}

\begin{equation}
	x = \frac{a}{n_0 \; \alpha} u
\end{equation}

Inverting the substitution and doing some rewriting yields the following equation:

\begin{equation}
	x = \frac{a}{n_0 \; \alpha} arcsin \left( \frac{r \; n_0 \; \alpha}{\sqrt{n_0^2-a^2}} \right)
\end{equation}

\begin{equation}
	\sin \left( \frac{n_0 \; \alpha \; r}{a} \right) =  \frac{r n_0 \alpha}{\sqrt{n_0^2-a^2}}
\end{equation}

\begin{equation}
	r(x) = \frac{\sqrt{n_0^2-a^2}}{n_0 \alpha} \sin \left( \frac{n_0 \; \alpha \; r}{a} \right) 
\end{equation}

The 
Since there are no discontinuities in $n(r)$ and $dn(r)/dr$, we would expect that there are also no discontinuities in $r(x)$ and $dr(x)/dx$. Since $r = \mid y \mid$ for the light beam in the $\vartheta _{x,y}$ plane. The only possible solution of $y(x)$ is the following:

\begin{equation}
	y(x) = \frac{\sqrt{n_0^2-a^2}}{n_0 \alpha} \sin \left( \frac{n_0 \; \alpha \; r}{a}\right) 
\end{equation}







\end{document}
